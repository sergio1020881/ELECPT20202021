\documentclass[titlepage, a4paper, 10pt, reqno, openany]{report}
\input{./input/PREAMBLE}
\begin{document}
\renewcommand\thesection{\arabic{section}}
\renewcommand\thesubsection{\thesection.\arabic{subsection}}
\renewcommand\thesubsubsection{\thesection.\thesubsection.\arabic{subsubsection}}
\pagestyle{plain}%plain headings empty
%%%%%%%%%%%%%%%%%%%%%%%%%%%%%%%%%%%%%%%%%%%%%%%%%%%%%%%%%%%%%%%%%%%%%%%%%%%%
\chapter*{Trabalho 3 Prep}
{\bf Nome: Sérgio Santos} \\
\hspace*{0.51cm}{\bf nº: 1020881}\\
\section{Esquema}
\begin{figure}[H]
	\centering
	\includegraphics[scale=0.7]{./image/esquema.jpg}\\
	\caption{Esquema}
\end{figure}
Não sei se devo acrescentar um drive na saída do sinal controlado, caso sim posso por um circuito lógico buffer. 
\newpage
%%%%%%%%%%%%%%%%%%%%%%%%%%%%%%%%%%%%%%%%%%%%%%%%%%%%%%%%%%%%%%%%%%%%%%%%%%%
\section{Material}
\begin{minipage}[t]{.60\linewidth}
	%	\quad List 1:
	\begin{itemize}
		\setlength\itemsep{-0.5em}
		\item Resistencias \\
		1/4 Watt, varias.
		\item Potenciometro multivolta \\
		100Kohm e 1Kohm
		\item Condensador \\
		10nF, 100nF  \\
		2x Electrolitico 10uF
		\item Diodo 1N4148 \\
		\item Fusivel 800mA \\
		\item Led Verde \\
	\end{itemize}
\end{minipage}
\begin{minipage}[t]{.31\linewidth}
	%	\quad List 2:
	\begin{itemize}
		\setlength\itemsep{-0.5em}
		\item TL084 \\
		Quand Opamp Chip \\
		14 Pinos \\
		\item Ficha Alimentação \\
	\end{itemize}
\end{minipage}\\
\\
Circuito Funciona de 10 Volt até 19 Volt testado em bancada, sendo necessario adjuste fino para calibração pelo Potenciómetro de  1Kohm. 
Pode se sempre consultar os {\it datasheets} dos componntes.\\

Ver Imagem do PCB caso duvidas. \\
\section{Simulação}
%\begin{comment}	
\begin{figure}[H]
	\centering
	\includegraphics[scale=0.15]{./image/placa_1.jpg}\\
	\caption{Top View}
\end{figure}
\begin{figure}[H]
	\centering
	\includegraphics[scale=0.15]{./image/placa_2.jpg}\\
	\caption{Back View}
\end{figure}\par
%%%%%%%%%%%%%%%%%%%%%%%%%%%%%%%%%%%%%%%%%%%%%%%%%%%%%%%%%%%%%%%%%%%%%%%%%%%
\section{Simulação}
%\begin{comment}	
\begin{figure}[H]
	\centering
	\includegraphics[scale=0.15]{./image/quadrado.jpg}\\
	\caption{Onda Quadrada}
\end{figure}
\begin{figure}[H]
	\centering
	\includegraphics[scale=0.15]{./image/triangulo.jpg}\\
	\caption{Onda Triangular}
\end{figure}\par
\newpage
%%%%%%%%%%%%%%%%%%%%%%%%%%%%%%%%%%%%%%%%%%%%%%%%%%%%%%%%%%%%%%%%%%%%%%%%%%%
\newpage
\footnote{Apontamentos}
%
	\end{document}
%%%EOF%%%
